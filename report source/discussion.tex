\section{Discussion}\label{sec:discussion}
In this section we will discuss the implications that the results of our experiment have, and the interpretation of them.

As will be treated in \autoref{sec:threats}, the results collected are not as good as we might hope, as the data have to be derived from other parameters, which harms the validity of this research.

Having said that, it might be interesting to dive deeper in this matter, provided that better ways of measuring energy consumption can be found. Also, when testing more applications, the data might give a better insight in the power consumption of native mobile applications and mobile web applications.

\textbf{RQ1 - How does energy efficiency differ between native mobile applications and their mobile web counterparts?}

Based on the results we concluded, that the energy consumption differs between native mobile applications and mobile web applications. This implies that users might save some battery life by using either native mobile applications or mobile web applications only. Our results overall point to a slight difference in the benefit of native mobile applications. However, because the number of results is very limited, more research is needed to decide if this conclusion holds with a larger number of applications.

Another uncertain part is the question whether or not a certain category of applications can be found that does deviate from the general trend, together with for example Google Maps or YouTube (the former more than the latter). 

\textbf{RQ2 - How does the device type affect the difference in energy consumption between native mobile applications and their mobile web counterparts?}

For Question 2, there is only one device in our experiment OnePlus 2, we could not test the affection of energy consumption between native mobile applications and their web application based on different device types. We could do deeper research in the future.
% Not sure if it is ok saying this way.