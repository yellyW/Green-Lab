\section{Conclusions}\label{sec:conclusions}
The goal of our paper was to evaluate energy consumption of native and web applications. In this paper we were able to answer \textbf{RQ1 - How does energy efficiency differ between native mobile applications and their mobile web counterparts?} in an experiment including 5 mobile applications in 2 variants: native and web. Results from statistical analysis showed that there is indeed a difference. Native variants of Google Maps and YouTube consumed more energy than the web variants. Facebook, Instagram, and Twitter had the opposite result.  What is more, due to the lack of time and resources, we could not research \textbf{RQ2 - How does the device type (high-end vs low-end) affect the difference in energy consumption between native mobile applications and their mobile web counterparts?} in our experiment.%should we call the numbers of the research questions here?

There are, however, several important shortcomings of our research that can be addresses in later projects. First of all, we were not able to use precise power consumption measurements. Secondly, our experiment includes only 5 of the most downloaded applications from Google Play Store. Thirdly, as we used only WiFi, we do not know how other network conditions affect the result. Finally, our focus was on the Android ecosystem with Google Chrome, and we cannot draw any conclusions about other ecosystems, e.g. iOS. All of those topics are good starting points for future experiments. Also, another extension of this experiment could be an experiment to find out why there is a difference in energy consumption between native mobile applications and mobile web applications.
 
%One brief paragraph for summarizing the main findings of the report.

%One brief paragraph about the possible extensions of the performed experiment (imagine that other 3 teams will be assigned to the extension of your experiment). 