\section{Introduction}\label{sec:intro}
\begin{figure*}[ht]
  \centering
  \subfloat[Facebook native application]{\includegraphics[height=0.4\textheight]{AppVSweb/figures/facebook_native.png}\label{fig:facebooknative}}
  \qquad
  \subfloat[Facebook web application]{\includegraphics[height=0.4\textheight]{AppVSweb/figures/facebook_web.png}\label{fig:facebookweb}}
  \caption[Facebook native vs Facebook web]{Facebook as an native \textcolor{blue}{(a)} versus Facebook as an web application \textcolor{blue}{(b)}}
  \label{fig:Facebook}
\end{figure*}

%<<<<<<<<<<<<<<<<<<<<<<<<<<<<TO ADD IN INTRODUCTION>>>>>>>>>>>>>>>>>>>>>>>>>>>>
% Deep   description   and critical thinking, evaluation of alternatives, etc.

Around 85\% of Italy's population today use smartphones. 80\% of the total time spent on smartphones is spent on native mobile applications, while only 20\% is spent on mobile Web applications \cite{MobileMark}. Furthermore, 92\% of people consider battery life as one of the significant factors when purchasing a new smartphone. 63\% of the users are unsatisfied with the battery life of their devices, which usually is one day or less, and 66\% would pay more money to extend their phones battery life \cite{BatteryLife_survey}.

Most native applications for the smartphone have a web application as well. For example, in figure \autoref{fig:Facebook} we see the native Facebook application versus the web version. Therefore, this report aims to evaluate whether it is possible to decrease energy consumption by using mobile web applications rather than native mobile applications.The idea that native mobile applications consume more energy is popular in mainstream media. It is relatively easy to find anecdotal proofs of this phenomenon \cite{itworld}. However, although  research has been done into the energy consumption of smartphones in general \cite{pathak2011fine}, mobile applications \cite{balasubramanian2009energy, li2014empirical, couto2014detecting, hao2013estimating} and more specific, into the energy consumption of web apps \cite{thiagarajan2012killed}, very little comparative research has been published between native mobile applications and mobile web applications. Our research will highlight the difference in energy consumption between native and web applications which will provide insight to users on which and to what extent applications types drains their battery, and thus, if they can save battery life by using mobile web applications rather than native mobile applications or the other way around.

An experiment with mobile web applications or native phone applications as independent variable and energy consumption as dependent variable will be carried out. \textit{If we have enough time}, device type (high-end or low-end) as a second independent variable will be added. Five to twenty pairs of the most popular applications will be evaluated using 1 \textit{or more, depending on time}, mobile phones in a controlled environment. To do so,common user actions for each application will be applied on each experimental trial.

%SUGGESTION BY ROBERT: For the critical note, we could make a small remark about using one browser instead of multiple, like this:
A note of caution has to be added, because the experiment is executed with one specific browser (\textit{Google Chrome}), it might be that an experiment with other browsers will yield different results than this research. However, using Chrome, this experiment do cover around 60\% of all browser users worldwide. \cite{Chromestats}